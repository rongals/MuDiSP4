
%% bare_conf.tex 
%% V1.1
%% 2002/08/13
%% by Michael Shell
%% mshell@ece.gatech.edu
%% 
%% NOTE: This text file uses UNIX line feed conventions. When (human)
%% reading this file on other platforms, you may have to use a text
%% editor that can handle lines terminated by the UNIX line feed
%% character (0x0A).
%% 
%% This is a skeleton file demonstrating the use of IEEEtran.cls 
%% (requires IEEEtran.cls version 1.6 or later) with an IEEE conference paper.
%% 
%% Support sites:
%% http://www.ieee.org
%% and/or
%% http://www.ctan.org/tex-archive/macros/latex/contrib/supported/IEEEtran/ 
%%
%% This code is offered as-is - no warranty - user assumes all risk.
%% Free to use, distribute and modify.

% *** Authors should verify (and, if needed, correct) their LaTeX system  ***
% *** with the testflow diagnostic prior to trusting their LaTeX platform ***
% *** with production work. IEEE's font choices can trigger bugs that do  ***
% *** not appear when using other class files.                            ***
% Testflow can be obtained at:
% http://www.ctan.org/tex-archive/macros/latex/contrib/supported/IEEEtran/testflow


% Note that the a4paper option is mainly intended so that authors in
% countries using A4 can easily print to A4 and see how their papers will
% look in print. Authors are encouraged to use U.S. letter paper when 
% submitting to IEEE. Use the testflow package mentioned above to verify
% correct handling of both paper sizes by the author's LaTeX system.
%
% Also note that the "draftcls", not "draft", option should be used if
% it is desired that the figures are to be displayed in draft mode.
\documentclass[conference]{IEEEtran}
% If the IEEEtran.cls has not been installed into the LaTeX system files, 
% manually specify the path to it:
% \documentclass[conference]{../sty/IEEEtran} 


% some very useful LaTeX packages include:

%\usepackage{cite}      % Written by Donald Arseneau
                        % V1.6 and later of IEEEtran pre-defines the format
                        % of the cite.sty package \cite{} output to follow
                        % that of IEEE. Loading the cite package will
                        % result in citation numbers being automatically
                        % sorted and properly "ranged". i.e.,
                        % [1], [9], [2], [7], [5], [6]
                        % (without using cite.sty)
                        % will become:
                        % [1], [2], [5]--[7], [9] (using cite.sty)
                        % cite.sty's \cite will automatically add leading
                        % space, if needed. Use cite.sty's noadjust option
                        % (cite.sty V3.8 and later) if you want to turn this
                        % off. cite.sty is already installed on most LaTeX
                        % systems. The latest version can be obtained at:
                        % http://www.ctan.org/tex-archive/macros/latex/contrib/supported/cite/

\usepackage{graphicx}  % Written by David Carlisle and Sebastian Rahtz
                        % Required if you want graphics, photos, etc.
                        % graphicx.sty is already installed on most LaTeX
                        % systems. The latest version and documentation can
                        % be obtained at:
                        % http://www.ctan.org/tex-archive/macros/latex/required/graphics/
                        % Another good source of documentation is "Using
                        % Imported Graphics in LaTeX2e" by Keith Reckdahl
                        % which can be found as esplatex.ps and epslatex.pdf
                        % at: http://www.ctan.org/tex-archive/info/

%\usepackage{psfrag}    % Written by Craig Barratt, Michael C. Grant,
                        % and David Carlisle
                        % This package allows you to substitute LaTeX
                        % commands for text in imported EPS graphic files.
                        % In this way, LaTeX symbols can be placed into
                        % graphics that have been generated by other
                        % applications. You must use latex->dvips->ps2pdf
                        % workflow (not direct pdf output from pdflatex) if
                        % you wish to use this capability because it works
                        % via some PostScript tricks. Alternatively, the
                        % graphics could be processed as separate files via
                        % psfrag and dvips, then converted to PDF for
                        % inclusion in the main file which uses pdflatex.
                        % Docs are in "The PSfrag System" by Michael C. Grant
                        % and David Carlisle. There is also some information 
                        % about using psfrag in "Using Imported Graphics in
                        % LaTeX2e" by Keith Reckdahl which documents the
                        % graphicx package (see above). The psfrag package
                        % and documentation can be obtained at:
                        % http://www.ctan.org/tex-archive/macros/latex/contrib/supported/psfrag/

%\usepackage{subfigure} % Written by Steven Douglas Cochran
                        % This package makes it easy to put subfigures
                        % in your figures. i.e. "figure 1a and 1b"
                        % Docs are in "Using Imported Graphics in LaTeX2e"
                        % by Keith Reckdahl which also documents the graphicx
                        % package (see above). subfigure.sty is already
                        % installed on most LaTeX systems. The latest version
                        % and documentation can be obtained at:
                        % http://www.ctan.org/tex-archive/macros/latex/contrib/supported/subfigure/

%\usepackage{url}       % Written by Donald Arseneau
                        % Provides better support for handling and breaking
                        % URLs. url.sty is already installed on most LaTeX
                        % systems. The latest version can be obtained at:
                        % http://www.ctan.org/tex-archive/macros/latex/contrib/other/misc/
                        % Read the url.sty source comments for usage information.

%\usepackage{stfloats}  % Written by Sigitas Tolusis
                        % Gives LaTeX2e the ability to do double column
                        % floats at the bottom of the page as well as the top.
                        % (e.g., "\begin{figure*}[!b]" is not normally
                        % possible in LaTeX2e). This is an invasive package
                        % which rewrites many portions of the LaTeX2e output
                        % routines. It may not work with other packages that
                        % modify the LaTeX2e output routine and/or with other
                        % versions of LaTeX. The latest version and
                        % documentation can be obtained at:
                        % http://www.ctan.org/tex-archive/macros/latex/contrib/supported/sttools/
                        % Documentation is contained in the stfloats.sty
                        % comments as well as in the presfull.pdf file.
                        % Do not use the stfloats baselinefloat ability as
                        % IEEE does not allow \baselineskip to stretch.
                        % Authors submitting work to the IEEE should note
                        % that IEEE rarely uses double column equations and
                        % that authors should try to avoid such use.
                        % Do not be tempted to use the cuted.sty or
                        % midfloat.sty package (by the same author) as IEEE
                        % does not format its papers in such ways.

\usepackage{amsmath}   % From the American Mathematical Society
                        % A popular package that provides many helpful commands
                        % for dealing with mathematics. Note that the AMSmath
                        % package sets \interdisplaylinepenalty to 10000 thus
                        % preventing page breaks from occurring within multiline
                        % equations. Use:
\interdisplaylinepenalty=2500
                        % after loading amsmath to restore such page breaks
                        % as IEEEtran.cls normally does. amsmath.sty is already
                        % installed on most LaTeX systems. The latest version
                        % and documentation can be obtained at:
                        % http://www.ctan.org/tex-archive/macros/latex/required/amslatex/math/



% Other popular packages for formatting tables and equations include:

% Frank Mittelbach's and David Carlisle's array.sty which improves the
% LaTeX2e array and tabular environments to provide better appearances and
% additional user controls. Array.sty is already installed on most systems.
% The latest version and documentation can be obtained at:
% http://www.ctan.org/tex-archive/macros/latex/required/tools/

% Mark Wooding's extremely powerful MDW tools, especially mdwmath.sty and
% mdwtab.sty which are used to format equations and tables, respectively.
% The MDWtools set is already installed on most LaTeX systems. The lastest
% version and documentation is available at:
% http://www.ctan.org/tex-archive/macros/latex/contrib/supported/mdwtools/

% V1.6 of IEEEtran contains the IEEEeqnarray family of commands that can
% be used to generate multiline equations as well as matrices, tables, etc.


% Also of notable interest:

% Scott Pakin's eqparbox package for creating (automatically sized) equal
% width boxes. Available:
% http://www.ctan.org/tex-archive/macros/latex/contrib/supported/eqparbox/




% *** Do not adjust lengths that control margins, column widths, etc. ***
% *** Do not use packages that alter fonts (such as pslatex).         ***
% There should be no need to do such things with IEEEtran.cls V1.6 and later.


% correct bad hyphenation here
\hyphenation{op-tical net-works semi-conduc-tor IEEEtran}

\newcommand{\E}{\mathrm{E}}
\newcommand{\W}{\mathrm{W}}

\begin{document}

% paper title
\title{Carrier Allocation Strategy for MC-CDMA uplink system}


% author names and affiliations
% use a multiple column layout for up to three different
% affiliations
\author{\authorblockN{Luca Simone Ronga}
\authorblockA{Italian National Consortium\\for Communications\\
University of Florence\\
Florence, 50139 Italy\\
Email: luca.ronga@cnit.it}
\and
\authorblockN{Enrico Del Re}
\authorblockA{Department of Electronics \\
and Telecommunications\\
University of Florence\\
Florence, 50139 Italy\\
Email: delre@lenst.det.unifi.it}
}


% avoiding spaces at the end of the author lines is not a problem with
% conference papers because we don't use \thanks or \IEEEmembership


% for over three affiliations, or if they all won't fit within the width
% of the page, use this alternative format:
% 
%\author{\authorblockN{Michael Shell\authorrefmark{1},
%Homer Simpson\authorrefmark{2},
%James Kirk\authorrefmark{3}, 
%Montgomery Scott\authorrefmark{3} and
%Eldon Tyrell\authorrefmark{4}}
%\authorblockA{\authorrefmark{1}School of Electrical and Computer Engineering\\
%Georgia Institute of Technology,
%Atlanta, Georgia 30332--0250\\ Email: mshell@ece.gatech.edu}
%\authorblockA{\authorrefmark{2}Twentieth Century Fox, Springfield, USA\\
%Email: homer@thesimpsons.com}
%\authorblockA{\authorrefmark{3}Starfleet Academy, San Francisco, California 96678-2391\\
%Telephone: (800) 555--1212, Fax: (888) 555--1212}
%\authorblockA{\authorrefmark{4}Tyrell Inc., 123 Replicant Street, Los Angeles, California 90210--4321}}



% use only for invited papers
%\specialpapernotice{(Invited Paper)}

% make the title area
\maketitle

\begin{abstract}
The abstract goes here.
\end{abstract}

% no key words

\section{Introduction}



%\section{The Noise-Loop Transmission Chain}
%
%In the analized system there are two terminals exchanging information: terminal no.~1 and terminal no.~2.
%Two different channels are considered: one for the link from terminal 1 to terminal 2 and one for the reverse link. The two channels are considered on different frequency bands and the thermal noise on one link is considered uncorrelated with the other. Each link is modeled as a conventional AWGN channel. 
%
%\subsection{Symbols in the paper}
%The following symbols have been adopted in the paper:
%\begin{description}
%	\item[$b_i$] binary antipodal information signal originating form terminal $i$,
%	\item[$n_i(t)$] Gaussian, white, continuos time random processes modelling the received noise at the receiver on terminal $i$, characterized by zero mean and variance $\sigma_n^2$,
%	\item[$\alpha_i$] global link gain for the signal generated from terminal $i$. It includes transmission gain, path loss and channel response. It is also supposed to be known by the receivers,
%	\item[$\tau_p$] propagation delay for the channel. It is assumed without loss of generality that both forward (1 to 2) and reverse (2 to 1) links have the same delay,
%	\item[$y_i(t)$] baseband received signal available at the terminal $i$
%\end{description}
%
%\subsection{Transmission Chain}
%
%In this simple transmission system the terminal operations are described in the fig.~\ref{fig:term}. The signal from the inbound channel is modulated by the information and re-transmitted on the outbound channel. The reception is obtained by extracting the sign of the $2\tau_p$-delayed autocorrelation term of the incoming signal, multiplied by the own informative bit. The process is explained in details in the following sections.
%%
%%\begin{figure}
%%\centering
%%\includegraphics[width=\columnwidth]{fig_term.pdf}
%%\caption{Noise-Loop scheme}
%%\label{fig:term}
%%\end{figure}
%
%\subsection{Stationary signal analysys in unlimited bandwidth}
%In this section a stationary condition on the system inputs is analysed\footnote{By stationary we intend a constant behavior over time of the information bits $b_i$, of the link gains and of the statistic parameters of the random processes involved}.
%Due to the additive nature of the model, the received signals $y_1(t)$ available at terminal 1 is defined by the following series:
%
%\begin{multline}
%	y_1(t)= \sum_{j=0}^{\infty}(b_1 b_2 \alpha_1 \alpha_2)^j n_1(t-2 j \tau_p) + \\
%	\sum_{j=0}^{\infty}(b_1 b_2 \alpha_1 \alpha_2)^j b_2 \alpha_2 n_2(t-(2 j+1)\tau_p)
%	\label{eq:rcv}
%\end{multline}
%
%An analogue expression can be obtained for $y_2(t)$ simply exchanging the subscript $1$ and $2$ in \eqref{eq:rcv}.
%
%The first term of \eqref{eq:rcv} represents the recursive contribution of the received noise $n_1(t)$ through the transmission loop. The second series on the other hand, is due to the injection of the noise process $n_2(t)$ by the other terminal. We also note that each term appears with a different delay on the received signal. 
%
%If the noise processes $n_i(t)$ are white on a unlimited bandwidth, then:
%
%\begin{equation}
%\E[n_i(t)n_j(t-\tau)]=
%\begin{cases}
%\begin{array}[c]{ll}
%\delta(\tau)\sigma_n^2 & i=j \\
%0 & i \ne j
%\end{array}
%\end{cases}
%\label{eq:noisecond}
%\end{equation}
%
%The structure of the signal in \eqref{eq:rcv} draw our attention in the shifted correlation term 
%\begin{equation}
%y_1(t-2\tau_p)y_1(t)
%\label{eq:acor}
%\end{equation}
%
%By resorting the terms obtained by the expansion of the above expression we obtain:
%\begin{multline}
%	\E[y_1(t-2\tau_p) y_1(t)] = \\
%	b_1 b_2 \sigma_n^2 ( 1 + \alpha_2^2 ) 
%		\sum_{j=0}^{\infty}(\alpha_1 \alpha_2)^{2j+1} \\
%	+ \E[\text{residual cross correlation terms}]		
%	\label{eq:aveacor}
%\end{multline}
%
%The last term in \eqref{eq:aveacor} is null for an ideal AWGN channel, so the described autocorrelation term is dominated by the information bearing $b_1 b_2$ term, weighted by a term which is constant in the stationary case. The term contains the information of both terminals. Since we are interested in the information from terminal no.~2 only, a post-multiplication by $b_1$ is performed. The receiver at terminal~1 is depicted in figure~\ref{fig:receiver}.
%
%%\begin{figure}
%%\centering
%%\includegraphics[width=\columnwidth]{fig_receiver.pdf}
%%\caption{Receiver scheme for terminal 1}
%%\label{fig:receiver}
%%\end{figure}
%
%%%%%%
%%%%%%
%%%%%%
%
%\section{Performance Analysis on ideal AWGN Channel}
%
%The term in \ref{eq:acor},  represents the instantaneous decision metric for the mutual information term to be estimated: $\hat{b_1} \hat{b_2}$. The performance of the receiver in terms of bit error probability is related to the first and second order statistics of \ref{eq:acor}. The distribution of the unpredictable noise process, however, is no longer Gaussian.
%
%\subsection{Auto-Correlation Noise}
%
%This specific case fall into the more general class of the resulting noise process derived by a product operation on two Gaussian processes. The application of the statistics theory bring to the integral formulation of the pdf resulting from the product of two Gaussian pdf. Let $x$ and $y$ be two Gaussian zero-mean processes with unitary variance, whose pdf's are:
%\begin{equation}
%	p_X(x)= \frac{1}{\sqrt{2 \pi}}e^{-\frac{x^2}{2}} \qquad
%	p_Y(y)= \frac{1}{\sqrt{2 \pi}}e^{-\frac{y^2}{2}}
%\end{equation}
%The pdf of the stochastic process $z=x y$ is a zero-order modified Bessel function: 
%\begin{equation}
%	p_Z(z)= \frac{1}{2 \pi} \int_{-\infty}^{\infty} \frac{e^{-\frac{z^2}{2 y^2}-\frac{y^2}{2}}}{|y|} \mathrm{d}y
%	= \frac{1}{\pi} J_0(|z|)
%	\label{eq:prod2gauss}
%\end{equation}
%
%where $J_0(\cdot)$ is the modified Bessel function of order zero. The distribution in \eqref{eq:prod2gauss} is shown in the figure~\ref{fig:prodpdf}.
%
%%\begin{figure}
%%\centering
%%\includegraphics[width=\columnwidth]{prodpdf.pdf}
%%\caption{Product of two Gaussian processes - pdf}
%%\label{fig:prodpdf}
%%\end{figure}
%
%The cumulative probability function in the positive semi-axis is also useful to the computation of the receiver performances. It has the same significance of the $Q$-function in the presence of Gaussian noise. It is defined for $x>0$ by:
%\begin{equation}
%	\text{($\W$-function)} \quad \W(x)= \frac{1}{\pi} \int_x^\infty J_0(z) \mathrm{d}z \quad (x>0)
%	\label{eq:wfunct}
%\end{equation}
%
%\subsection{Performance Evaluation}
%
%The decision term in \eqref{eq:acor} is characterized by a probability density function defined by \eqref{eq:prod2gauss} with a mean value of 
%\[\E[y_1(t-2\tau_p)y_1(t)]\]
%and a second order moment of
%\[var[y_1(t-2\tau_p)y_1(t)]\]
%For binary antipodal signaling the bit error probability is related to the proviously defined $\W$-function by:
%\begin{equation}
%P_e = \W\left( \sqrt{\frac{\E[y_1(t-2\tau_p)y_1(t)]^2}
%	{var[y_1(t-2\tau_p)y_1(t)]}}\right) = \W\left(\sqrt{\gamma_b} \right)
%\label{eq:pe01}
%\end{equation}
%where $\W()$ is the W-function in \eqref{eq:wfunct} and $var[]$ indicates the variance operator. The argument of \ref{eq:pe01}, $\gamma_b$,  can be expanded into:
%\begin{multline}
%	\gamma_b = \frac{\E[y_1(t-2\tau_p)y_1(t)]^2}
%	{var[y_1(t-2\tau_p)y_1(t)]}= \\
%	\frac{\sigma_n^4(1+\alpha_2^2)^2\left( \sum_{i}
%	(\alpha_1 \alpha_2)^{(2j+1)}\right)^2}
%	{\sigma_n^4\left[
%		(1+\alpha_2^2)^2 \sum_i \sum_j (\alpha_1^2 \alpha_2^2)^{j+l} + (1+\alpha_2^4) \sum_j (\alpha_1^2 \alpha_2^2)^{2j+1}
%		\right]} 
%	\label{eq:snr}
%\end{multline}
%Where all the series, unless explicitely defined, are intended ranging from zero to infinity. The $\gamma_b$ term does not depends on the noise variance. The observation can be explained with the consideration that the higher is the noise floor, the stronger will be the injection of modulated signal in the loop. The signal-to-noise ratio depends from the closed loop total gain which is lesser than unity for stability. In figure~\ref{fig:gamma} is plotted the value of $\gamma_b$ function of $\alpha=\alpha_1=\alpha_2$. 
%
%%\begin{figure}
%%\centering
%%\includegraphics[width=\columnwidth]{gamma.pdf}
%%\caption{$\gamma_b$ function of $\alpha$}
%%\label{fig:gamma}
%%\end{figure}
%
%The probability of error for the proposed trasmission system can be evaluated as a function of the signal-to-noise ratio ($\gamma_b$) numerically. The figure~\ref{fig:pegamma} shows the $P_e$ values vs. $SNR$ for the proposed thermal noise modulation and a traditional BPSK modulation. 
%
%%\begin{figure}[b]
%%\centering
%%\includegraphics[width=\columnwidth]{pegamma.pdf}
%%\caption{Bit error probability vs. SNR for noise modulation and traditional carrier modulation}
%%\label{fig:pegamma}
%%\end{figure}
%
%\subsubsection{Subsubsection Heading Here}
%Subsubsection text here.
%
%% Reminder: the "draftcls", not "draft", class option should be used if
%% it is desired that the figures are to be displayed while in draft mode.
%
%% An example of a floating figure using the graphicx package.
%% Note that \label must occur AFTER (or within) \caption.
%% For figures, \caption should occur after the \includegraphics.
%%
%%\begin{figure}
%%\centering
%%\includegraphics[width=2.5in]{myfigure.eps}
%%\caption{Simulation Results}
%%\label{fig_sim}
%%\end{figure}
%
%
%% An example of a double column floating figure using two subfigures.
%%(The subfigure.sty package must be loaded for this to work.)
%% The subfigure \label commands are set within each subfigure command, the
%% \label for the overall fgure must come after \caption.
%% \hfil must be used as a separator to get equal spacing
%%
%%\begin{figure*}
%%\centerline{\subfigure[Case I]{\includegraphics[width=2.5in]{subfigcase1.eps}
%%\label{fig_first_case}}
%%\hfil
%%\subfigure[Case II]{\includegraphics[width=2.5in]{subfigcase2.eps}
%%\label{fig_second_case}}}
%%\caption{Simulation results}
%%\label{fig_sim}
%%\end{figure*}
%
%
%
%% An example of a floating table. Note that, for IEEE style tables, the 
%% \caption command should come BEFORE the table.Table text will default to
%% \footnotesize as IEEE normally uses this smaller font for tables.
%% The \label must come after \caption as always.
%%
%%\begin{table}
%%% increase table row spacing, adjust to taste
%%\renewcommand{\arraystretch}{1.3}
%%\caption{An Example of a Table}
%%\label{table_example}
%%\begin{center}
%%% The array package and the MDW tools package offers better commands
%%% for making tables than plain LaTeX2e's tabular which is used here.
%%\begin{tabular}{|c||c|}
%%\hline
%%One & Two\\
%%\hline
%%Three & Four\\
%%\hline
%%\end{tabular}
%%\end{center}
%%\end{table}
%
%
%\section{Conclusion}
%The conclusion goes here.
%
%% conference papers do not normally have an appendix
%\section*{Appendix A}
%
%The partial expansion of the signal at terminal~1 is
%
%\begin{equation*}
%\begin{split}
%y_1(t) &= n_1(t) \\
%&+ b_2 \alpha_2 n_2(t - \tau_p)\\
%&+ b_1 b_2 \alpha_1 \alpha_2 n_1(t - 2 \tau_p)\\
%&+ b_1 b_2^2 \alpha_1 \alpha_2^2 n_2(t - 3 \tau_p)\\
%&+ b_1^2 b_2^2 \alpha_1^2 \alpha_2^2 n_1(t - 4 \tau_p)\\
%&+ b_1^2 b_2^3 \alpha_1^2 \alpha_2^3 n_2(t - 5 \tau_p)\\
%&+ b_1^3 b_2^3 \alpha_1^3 \alpha_2^3 n_1(t - 6 \tau_p)\\
%&+ b_1^3 b_2^4 \alpha_1^3 \alpha_2^4 n_2(t - 7 \tau_p) + ... 
%\end{split}
%\end{equation*}
%
%and the $2 \tau$-delayed version:
%
%\begin{equation*}
%\begin{split}
%y_1(t- 2 \tau_p) &= n_1(t-2 \tau_p) \\
%&+ b_2 \alpha_2 n_2(t - 3 \tau_p)\\
%&+ b_1 b_2 \alpha_1 \alpha_2 n_1(t - 4 \tau_p)\\
%&+ b_1 b_2^2 \alpha_1 \alpha_2^2 n_2(t - 5 \tau_p)\\
%&+ b_1^2 b_2^2 \alpha_1^2 \alpha_2^2 n_1(t - 6 \tau_p)\\
%&+ b_1^2 b_2^3 \alpha_1^2 \alpha_2^3 n_2(t - 7 \tau_p)\\
%&+ b_1^3 b_2^3 \alpha_1^3 \alpha_2^3 n_1(t - 8 \tau_p)\\
%&+ b_1^3 b_2^4 \alpha_1^3 \alpha_2^4 n_2(t - 9 \tau_p) + ... 
%\end{split}
%\end{equation*}
%
%The instantaneus $2 \tau$-delayed auto-correlation term is given by :
%
%\begin{equation*}
%\begin{split}
%y_1(t)y_1(t- 2 \tau_p) &= \\
%&+\mathbf{b_1 b_2 \alpha_1 \alpha_2 n_1(t - 2 \tau_p) n_1(t-2 \tau_p)} \\
%&+\mathbf{b_1 b_2^3 \alpha_1 \alpha_2^3 n_1(t - 3 \tau_p) n_1(t-3 \tau_p)} \\
%&+\mathbf{b_1^3 b_2^3 \alpha_1^3 \alpha_2^3 n_1(t - 4 \tau_p) n_1(t-4 \tau_p)} \\
%&+ ...
%\end{split}
%\end{equation*}
%
%\section{Appendix B}
%
%% use section* for acknowledgement
%\section*{Acknowledgment}
%% optional entry into table of contents (if used)
%%\addcontentsline{toc}{section}{Acknowledgment}
%The authors would like to thank...
%
%% trigger a \newpage just before the given reference
%% number - used to balance the columns on the last page
%% adjust value as needed - may need to be readjusted if
%% the document is modified later
%%\IEEEtriggeratref{8}
%% The "triggered" command can be changed if desired:
%%\IEEEtriggercmd{\enlargethispage{-5in}}
%
%% references section
%% NOTE: BibTeX documentation can be easily obtained at:
%% http://www.ctan.org/tex-archive/biblio/bibtex/contrib/doc/
%
%% can use a bibliography generated by BibTeX as a .bbl file
%% standard IEEE bibliography style from:
%% http://www.ctan.org/tex-archive/macros/latex/contrib/supported/IEEEtran/testflow/bibtex
%%\bibliographystyle{IEEEtran.bst}
%% argument is your BibTeX string definitions and bibliography database(s)
%%\bibliography{IEEEabrv,../bib/paper}
%%
%% <OR> manually copy in the resultant .bbl file
%% set second argument of \begin to the number of references
%% (used to reserve space for the reference number labels box)
%\begin{thebibliography}{1}
%
%\bibitem{IEEEhowto:kopka}
%H.~Kopka and P.~W. Daly, \emph{A Guide to {\LaTeX}}, 3rd~ed.\hskip 1em plus
%  0.5em minus 0.4em\relax Harlow, England: Addison-Wesley, 1999.
%
%\end{thebibliography}


% that's all folks
\end{document}


